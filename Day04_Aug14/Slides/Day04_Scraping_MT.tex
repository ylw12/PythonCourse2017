\documentclass[xcolor=x11names,compress]{beamer}

%% General document %%%%%%%%%%%%%%%%%%%%%%%%%%%%%%%%%%
\PassOptionsToPackage{table}{xcolor}
\usepackage{graphicx}
\usepackage{multirow,multicol}
\usepackage{amsmath}
\usepackage{mathpazo}
\usepackage{amsthm}
\usepackage{amssymb}
\usepackage{setspace}
\usepackage{hyperref}
\usepackage{array,colortbl,booktabs}
\usepackage{soul}
\usepackage{enumerate}
\usepackage{url}
\usepackage{verbatimbox}
\usepackage{fancyvrb}
\usepackage{dirtree}
\usepackage{tikz}
\usepackage[utf8]{inputenc}
\usepackage[T1]{fontenc}
\usetikzlibrary{positioning,shapes.misc}
\usetikzlibrary{decorations.fractals}
\usetikzlibrary{calc}
\usepackage[normalem]{ulem}
\useunder{\uline}{\ul}{}
\usetikzlibrary{arrows}
\usetikzlibrary{fit}
%\usepackage{color, colortbl}
\usepackage{etoolbox}
\makeatletter
\patchcmd{\slideentry}{\advance\beamer@xpos by1\relax}{}{}{}
\def\beamer@subsectionentry#1#2#3#4#5{\advance\beamer@xpos by1\relax}%
\makeatother
%%%%%%%%%%%%%%%%%%%%%%%%%%%%%%%%%%%%%%%%%%%%%%%%%%%%%%


%% Beamer Layout %%%%%%%%%%%%%%%%%%%%%%%%%%%%%%%%%%
\useoutertheme[subsection=false, shadow]{miniframes}
\useinnertheme{circles}
\usefonttheme{structurebold}
\usepackage{palatino}
\usepackage{tcolorbox}
\usepackage{lipsum}

%% COLOR DEFINITION %%%%%%%%%%%%%%%%%%%%%%%%%%%%%%%%
\definecolor{mygold}{RGB}{236,208,120}
\definecolor{brick}{RGB}{217,91,67}
\definecolor{myred}{RGB}{192,41,66}
\definecolor{redwine}{RGB}{102,0,102}
\definecolor{myaqua}{RGB}{83,119,122}

%%%%%%%%%%%%%%%%%%%%%%%%%%%%%%%%%%%%%%%%%%%%%%%%%%%%%%

%%% OTHER EXTRA STUFF%%%%%%%%%%%%%%%%%%%%%%%%%%%%%%%%
%% CHANGE COVER SLIDE %%
\makeatletter
    \newenvironment{coverframe}{
\setbeamercolor*{palette tertiary}{fg=myaqua,bg=myaqua} 
}
    {}
\makeatother

%% CHANGE COLOR OF BULLETS %%
\setbeamercolor{item}{fg=brick} % color of bullets
\setbeamercolor{subitem}{fg=mygold}

%% DEFINE TIKZ ITEMS %%
\tikzset{
    %Define standard arrow tip
    >=stealth',
    %Define style for boxes
    punkt/.style={
           rectangle,
           rounded corners,
           draw=black, very thick,
           text width=3em,
           minimum height=2em,
           text centered},
    % Define arrow style 1
    pil/.style={
           <-,
           thick,
           shorten <=2pt,
           shorten >=2pt,}
        % Define arrow style
           }
           
 %% Perpendiculas symbol
 \newcommand\independent{\protect\mathpalette{\protect\independenT}{\perp}}
\def\independenT#1#2{\mathrel{\rlap{$#1#2$}\mkern2mu{#1#2}}}

\newcommand{\ds}{\displaystyle}

\newcommand{\bv}{\begin{Verbatim}[numbers=left, baselinestretch=1,
    xleftmargin=.5in, xrightmargin=.1in, frame=single,
    rulecolor=\color{gray}]}

%%%%%%%%%%%%%%%%%%%%%%%%%%%%%%%%%%%%%%%%%%%%%%%%%%
\setbeamerfont{title like}{shape=\scshape}
\setbeamerfont{frametitle}{shape=\scshape}
\setbeamercolor{frametitle}{fg=myaqua!120,bg=white}
\setbeamercolor{title}{fg=myaqua!120}
%\setbeamertemplate{frametitle}{\vspace{6em}}

\setbeamercolor*{lower separation line head}{bg=redwine} 
\setbeamercolor*{normal text}{fg=black} 
\setbeamercolor*{alerted text}{fg=myred} 
\setbeamercolor*{example text}{fg=brick} 
\setbeamercolor*{structure}{fg=black} 
 
\setbeamercolor*{palette tertiary}{fg=black!80,bg=white} 
\setbeamercolor*{palette quaternary}{fg=black,bg=black!10} 

\renewcommand{\(}{\begin{columns}}
\renewcommand{\)}{\end{columns}}
\newcommand{\<}[1]{\begin{column}{#1}}
\renewcommand{\>}{\end{column}}

%%%%%%%%%%%%%%%%%%%%%%%%%%%%%%%%%%%%%%%%%%%%%%%%%%
\title[Scraping using Selenium and BeautifulSoup]{Python Class 4: Scrape the web!}
\author{Michelle Torres}
\date{August 14, 2016}

\begin{document}
\newcommand<>{\highlighton}[1]{%
  \alt#2{\structure{#1}}{{#1}}
}

\newcommand{\icon}[1]{\pgfimage[height=1em]{#1}}

%%%%%%%%%%%%%%%%%%%%%%%%%%%%%%%%%%%%%%%%%%%%%%%%%%
%%%%%%%%%%%%%%%%%%%%%%%%%%%%%%%%%%%%%%%%%%%%%%%%%%
%%%%%%%% START THE SLIDES %%%%%%%%%%%%%%%%%%%%%%%%
%%%%%%%%%%%%%%%%%%%%%%%%%%%%%%%%%%%%%%%%%%%%%%%%%%
%%%%%%%%%%%%%%%%%%%%%%%%%%%%%%%%%%%%%%%%%%%%%%%%%%

\begin{coverframe}
\begin{frame}
\titlepage	
\end{frame}
\end{coverframe}


\section[Outline]{}
\frame{\tableofcontents}

\section{Urllib and Beautiful Soup}
\frame{
\frametitle{Overview}
\begin{enumerate}
	\item Call the website and open it
	\item Extract the html code
	\item Retrieve information using the names of the nodes, tags, ids, etc.
	\item Store it in lists or directly to CSV files
\end{enumerate}
}

\begin{frame}[fragile]
\frametitle{Getting to know the page source}
\begin{verbatim}
<!DOCTYPE html>
<html>
<head>
<title>Page Title</title>
</head>
<body>

<h1>My First Heading</h1>
<p>My first paragraph.</p>

</body>
</html>
\end{verbatim}	
\end{frame}


\section{Selenium}
\frame{
\frametitle{Why do we care about another approach?}
\begin{itemize}[<+->]
	\item Beautiful Soup and Urllib are ``crawlers'' that operate in the ``background''.
	\item They are incredibly fast...
	\item ... but also easier to detect and block
	\item Plus, they sometimes a bit restrictive if you don't know the exact node or thing to extract
\end{itemize}
}

\frame{
\frametitle{Selenium}
\begin{itemize}[<+->]
	\item Selenium is a ``remote driver'' of your favorite browser
	\item Therefore, you can pretty much simulate behavior of a human ``surfing the web''
	\item With the right tricks, the likelihood of tracking and blocking your ``bot'' decreases.
	\item It also offers flexibility in terms of ``unknown'' items: you can even look by name of buttons in the page
	\item \textcolor{myred}{There are some downsides though...}
	\begin{itemize}
		\item It is slower
		\item It is dependent on your internet connection quality
		\item Sometimes it is *very* annoying!
	\end{itemize}
\end{itemize}
}

\section{Some tricks}
\frame{
\frametitle{Some tricks and advice to become a pro-scraper}

\begin{itemize}[<+->]
	\item Google Chrome is better to track nodes and page sources
	\item Inspect the source and get to know your document/website!
	\item Use the 'Copy Xpath' command if you're having troubles
	\item If you have a complex website, visit the websites that store java scripts. They give hints on how information is displayed
	\item Use time breaks to avoid being blocked
\end{itemize}
}


\end{document}
